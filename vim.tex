\section{general}%
\label{sec:general}

to replace, use s, eg to replace (g)lobally foo with bar:
\begin{verbatim}
	:%s/foo/bar/g
\end{verbatim}

to globally delete all lines containing a string use:
\begin{verbatim}
	:g/string/d
\end{verbatim}

to run a process on the current file (\%)in the shell and get the output back into the file do:
\begin{verbatim}
	:%!command
\end{verbatim}

to quickly sort a list, it can be useful to use linux's built in sort and keep unique values (-u flag):
\begin{verbatim}
	:%!sort -u
\end{verbatim}

to overwrite a file you forgot to open in sudo then:
\begin{verbatim}
	:w !sudo tee %
\end{verbatim}

normally to edit a single file :e is used, however vim will protest if you try to load up more than one file, instead use:
\begin{verbatim}
	:args /path/*
\end{verbatim}
\section{working locally on a remote file}%
\label{sec:working_locally_on_a_remote_file}

that's right, this shit is possible, thanks to scp. directly from vim you can just e it:
\begin{verbatim}
	:e scp://user@server//path/path/file.ext	
\end{verbatim}
or dive into it straight from terminal:
\begin{verbatim}
	$scp://remoteuser@server.tld//absolute/path/to/document
\end{verbatim}

\section*{windows}%
\label{sec:windows}

C-w-| zooms window fullscreen
C-w-= restores splits
