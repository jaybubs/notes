\section{General}%
\label{sec:general}

\subsection{Syntax}%
\label{sub:syntax}

You can use the built-in syntax checker straight from cli:
\begin{verbatim}
	$ php -l filename.php
\end{verbatim}

or even better if you're using vim, then straight off the bat, \% will feed the current file into the argument:
\begin{verbatim}
	:!php -l %
\end{verbatim}

\subsection{variables}%
\label{sub:variables}

In PHP, you have to declare the global variables you want to use in a function, they're not accessible by default. You have to have global \$f3; at the first line of the function in order for it to access the variable.
Passing \$f3 as an argument is syntactic sugar, it's a sweet shortcut for controller functions but it does basically the same.

\section{MVC}%
\label{sec:mvc}

The MVC model-view-controller is a standard webdev architectural pattern, that adheres to separation of concerns and reusability/modularity philosophies. The cycle is fairly straightforward, the user sees the view, makes a request, the controller then forwards the request to the appropriate model and the model then does whatever code needs to be done to give back a result.

\subsection{Model}%
\label{sub:model}

This is the central component really, the major chunk of code is most likely to sit here. It may be anything from editing data in a database, to performing calculations.

\subsection{Controller}%
\label{sub:controller}

This is the slim routing part. It takes input from the user, processes it, passes it onto the model and/or retrieves back values. It doesn't do any heavy lifting, it's more or less just the middle man.

\subsection{View}%
\label{sub:view}

This is the user-end interface, such as an app front or a web page with forms for the user to interact with and send values to the backend.
